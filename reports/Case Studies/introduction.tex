\chapter{Introduction}\label{ch:introduction}
Target systems should be grouped into two main categories:
 
\begin{itemize}
  \item Software ecosystems: "A software ecosystem is a set of actors functioning as a
unit and interacting with a shared market for software and services, together
with the relationships among them. These relationships are frequently
underpinned by a common technological platform or market and operate through the
exchange of information, resources and artifacts" \cite{jansen_sense_2009}.
\item Systems of Systems: "System of systems are large-scale
integrated systems that are heterogeneous and independently operable on their
own, but are networked together for a common goal" \cite{janishidi_system_2008}.
\end{itemize}                   	
                                                               	
From a pure technical perspective, the main difference  between the two group is
that a system of systems is a system whose parts are themselves operationally
independent systems that can be considered in isolation. A system of systems was
not designed as a whole, and perhaps could never have been, but nevertheless the
whole is more than the sum of its parts and may have emergent properties. In an
ecosystem, in contrast, many of its parts depend on one another and therefore
cannot function in isolation. A software ecosystem is a collection of software
projects which are developed and evolve together in the same environment.


