\chapter{Software Ecosystems}\label{ch:softwareEcosystems}
To this class belong software systems that are built on top of a technical
software platform by composing components developed by actors both internal and
external.

\begin{itemize}
  \item IDEs:
  \begin{itemize}
    \item Eclipse Ecosystem: The Eclipse ecosystem is the universe of Eclipse
    plugins together with the developers of these plugins. All different Eclipse plugins rely on a common underlying architecture, platform and set of libraries without which they are unable to function correctly. The community of plugin developers therefore shares the common goal of improving a complete integrated software development environment. Bug example:
    \begin{itemize}
      \item Description: Bugzilla queries gone after moving from Eclipse 3.5 to
      3.5.1.
      \item Link: https://bugs.eclipse.org/bugs/show\_bug.cgi?id=291334
      \item Root cause: ExtensionEventDispatcherJob sets a mutually exclusive rule on the
      thread then initializes tasks.ui which in turn tries to set a conflicting rule.
      \item Status: Postponed
    \end{itemize}
  \end{itemize}
  
  \item Programming Language Community:
  \begin{itemize}
    \item The CRAN Ecosystem (cran.r-project.org): The GNU R community shares the
    goal of creating a statistical computing environment. It achieves this through
    the Comprehensive R Archive Network (CRAN), a developer-centric ecosystem in which each community member can contribute packages implementing specific statistical analysis functions and visualization tools. A particularity of this ecosystem is that the community is mainly composed of end-users, that do not necessarily have a lot of development experience. CRAN has been experiencing maintainability problems due to the superlinearly growing number of packages and the limitations of R’s dependency versioning system. Bug example:
    \begin{itemize}
      \item Description: qchisq produced NaN when df=0.00991.
      \item Link: https://bugs.r-project.org/bugzilla3/show\_bug.cgi?id=14710
      \item Root cause: The replacement
      pgamma algorithm used from R 2.1.0 has an inadequate design. A solution is
      to revert where needed to the normal approximation used in R 2.0.0
      \item Status: Closed.
    \end{itemize}
  \end{itemize}
  \item Open source software foundation:
  \begin{itemize}
    \item Apache ecosystem: It is composed of 195 software projects
    spread in 23 different categories (eg, big-data, FTP, mobile, library, testing, XML) and
    developed by using a total of 29 programming languages. Bug example:
    \begin{itemize}
      \item Description: AreaReference(String) has been deprecated. This makes
      AreaPtg(String), Area3DPtg(String, externIdx), Area3DPxg(SheetIdentifier,
      String), Area3DPxg(int, SheetIdentifier, String) deprecated constructors.
      \item Link: https://bz.apache.org/bugzilla/show\_bug.cgi?id=58331
      \item Root cause: AreaReference(String) has been deprecated from bug 56328.
      \item Status: Closed.
    \end{itemize}
    \item Mozilla Bug example:
    \begin{itemize}
      \item Description: Browser Window turns black after pressing Press STRG
      + TAB
      \item Link: https://bugzilla.mozilla.org/show\_bug.cgi?id=660160\#c14
      \item Root cause:  Fox Tab incompatibility. This add-o was affected by the
      fix in bug 637204.
      \item Status: Closed.
    \end{itemize}
  \end{itemize}
  
  \item Mobile App Store:
  \begin{itemize}
    \item Android: It is an open source Linux-based operating system owned by
    Google and designed for touchscreen mobile devices such as smart-phones and
    tablet computers. Google supports mobile app development with
    its own operating system and associated Application
    Programming Interfaces (APIs). These APIs give developers
    access to features like location services, wi-fi connections,
    bluetooth functionality, and graphics. Bug example:
    \begin{itemize}
      \item Description: Without warning, upon loading activity or fragment
      containing multiple map fragments (or a little later, upon trying to
      interact with the map fragments),  the app will crash.
      
      
      \item Link: https://code.google.com/p/gmaps-api-issues/issues/detail?id=5100
      \item Root cause:  The Google Maps API is broken on Android 5.
      \item Status: Closed.
    \end{itemize}
  \end{itemize}
  
\end{itemize}
\section{Studies on Software Ecosystem} 

Dig and Johnson \cite{dig_role_2005} investigated the evolution of three
frameworks including Eclipse and the log4j library. They found out that most
(from 81\% to 97\%) of the application-breaking changes were comprised
by structural, behavior-preserving transformations (refactorings).

In a large scale study of the Smalltalk development communities, Robbes et al.
\cite{robbes_how_2012} found that only 14\% of deprecated methods produce
non-trivial API change effects in at least one client-side project; however,
these effects vary greatly in magnitude. On average, a single API deprecation
resulted in 5 broken projects, while the largest caused 79 projects
and 132 packages to break.

Kim et al. \cite{kim_empirical_2011} investigate the relationship between API
refactorings and bugs and find the bug fix rate is higher after API-level refactorings than the preceding
period, e.g., from 26.1\% to 30.3\% when examining 5 revisions
before and after API-level refactorings in Eclipse JDT. They also found that a
fix rate increase is often caused by mistakes in applying refactorings and
behavior modifying edits together.


An exploratory study of project inter-dependencies in the 
Apache ecosystem \cite{bavota_evolution_2013} shows that:
\begin{itemize}
  \item client projects tend to upgrade their dependencies when substantial changes in the projects they depend on are released,
including bug-fixing activities
\item the proportion
of source code of client projects impacted by changes in
the projects they depend on is quite limited, around 5\%.
However, there are specific dependencies, generally toward frameworks/libraries
(e.g.  Commons Logging) offering very wide, crosscutting services, that could strongly
impact the client project source code when a dependency is upgraded
\item 
\end{itemize} 


An empirical study of API stability and adoption in the android ecosystem
\cite{mcdonnell_empirical_2013} shows that:
\begin{itemize}
  \item Android APIs are evolving
at the rate of 115 API updates per month on average.
APIs related to hardware, user interface, and web are
evolving much faster than others.
  \item Around 25\% of all method and field references in the
client code use the Android APIs. However, application
developers are hesitant to adopt new APIs. On average,
28\% of Android API calls are lagging behind the latest
released version. 22\% of outdated APIs eventually
upgrade to use newer APIs; nevertheless it takes a
considerable amount of time, 14 months, on average
\item Fast evolving APIs are used
more by clients than slow evolving APIs. However, the pace of client update is
slower for fast evolving APIs.
Files which are changed to use new APIs are more
defect prone than files without API usage adaptation.
This may imply that developers avoid frequent upgrades
to unstable or rapidly evolving APIs.
\end{itemize}


