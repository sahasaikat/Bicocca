\chapter{Systems of Systems}\label{ch:SoS}
To this class belong any software systems arisen as result of integration of various operationally independent systems, even developed with different technologies and for diverse platforms. The integration is necessary to promote cooperation among these independent systems in order to provide more complex functions, which could not be provided by any system working separately. The services view is generally applied in this kind of software systems.
\begin{itemize}                                           	
\item Integrated Air Defense: The air defenses of modern military forces are
clear examples of systems-of- systems. An integrated air defense system is composed of a geographically dispersed network of semi-autonomous elements. These include surveillance radars, passive surveillance systems, missile launch batteries, missile tracking and control sites, airborne surveillance and tracking radars, fighter aircraft, and anti-aircraft artillery. All units are tied together by a communications network with command and control applied at local, regional, and national centers.
 
\item Airports: Independent systems for operation of each airline,
leasing/selling of resources, air traffic control of this and other airports; support services of catering, rail and roads. An airport is another complex system; however, the airport involves aircraft, support trucks, baggage-handling equipment, and many other systems that can and do operate independently of each other. For the airport to function, it needs to have the right mix of these independent systems, and these systems need to cooperate with each other.
 
\item e-Commerce: e-commerce involves a number of different independent systems
working together in order to reach the overall goal of sale and delivery. These systems perform functions that provide the virtual market place, the handling of payment transactions, management of inventory and supply chains, as well as handling shipping and delivery
                                            	
\item Transport network: Transport networks are often composed of independently
owned and operated systems that are geographically distributed.  Transport network covers a wide range of potential applications of information and computer technology to road and transport networks. These run the gamut from improved public service vehicle communication to automated highways with robotically driven cars.
\end{itemize}


https://www.clementine-player.org

https://pidgin.im
