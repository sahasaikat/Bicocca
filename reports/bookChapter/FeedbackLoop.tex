\section{Feedback Loop}



Introduction on Feedback Loop\\
Picture of Feedback Loop\\
Brief description of each of the phases of Feedback Loop

Feedback loops also referred as autonomic control loop plays a major role in building self adaptive system. A feedback loop is a control loop where the output of the control system is feed as an input to adjust the difference between the expected output and the reference input. A set of reference input is feed to the controller which maintains the specified properties of the expected output. The output generated by the process is feed to the sensors to calculate the difference between the generated output and given input reference. The common phenomenon which may lead to this difference is due to the disturbances. The difference and the specified properties calculates the correction using set of algorithms mentioned in the controller to drive the process output to the reference input. Figure shows the basic set up of a generic control loop. Figure shows the general set-up of a feedback loop describing each of the components, where P is the process, C is the controller, d is the  


The four stages of a typical feedback loop are; collect, analyse, decide and act. Data and information about the system is collected by sensors which are stored in a model representing the state of the running system. The collected data is analysed to identify the symptoms. During planning phase the appropriate actions are triggered though a set of effector.

\paragraph*{Collect}
In the collection phase, sensors and probes collect information about the current state
of the system. The collected information is preprocessed using several data filtration steps.    

\paragraph*{Analyse}
In the analysis phase the collected data is analysed to infer the current state of the system, the amount of past data required to perform the analysis and the amount of data which is relevant for verification and validation.



\paragraph*{Decide}

The decision phase adapts the way the system can reach the desired state. There are several approaches like risk analysis and decision theory which are used to change the state. 

\paragraph*{Act}

The act phase executes the  decisions using effectors or actuators to act on the managed system 